\documentclass[lettersize,journal]{IEEEtran}
\usepackage{amsmath,amsfonts}
\usepackage{algorithmic}
\usepackage{algorithm}
\usepackage{array}
\usepackage[caption=false,font=normalsize,labelfont=sf,textfont=sf]{subfig}
\usepackage{textcomp}
\usepackage{stfloats}
\usepackage{url}
\usepackage{verbatim}
\usepackage{graphicx}
\usepackage{lipsum}
\usepackage{cite}
\hyphenation{op-tical net-works semi-conduc-tor IEEE-Xplore}
% updated with editorial comments 8/9/2021

\begin{document}

\title{DNN Model Placement for Edge Intelligence}

\author{Convex Optimization 2 Course Project, Nima Samadi, Mohammad Javad Mohammadi}

% The paper headers
%\markboth{Journal of \LaTeX\ Class Files,~Vol.~14, No.~8, August~2021}%
%{Shell \MakeLowercase{\textit{et al.}}: A Sample Article Using IEEEtran.cls for IEEE Journals}

%\IEEEpubid{0000--0000/00\$00.00~\copyright~2021 IEEE}
% Remember, if you use this you must call \IEEEpubidadjcol in the second
% column for its text to clear the IEEEpubid mark.

\maketitle

\begin{abstract}

\end{abstract}

\begin{IEEEkeywords}
Edge intelligence, Model placement, Neural network, Convex optimization
\end{IEEEkeywords}

\section{Introduction}

Deep Neural Networks (DNNs) have become increasingly popular as the core machine learning technique in many fields, such as speech recognition, image classification, translation, language modeling, and video captioning. 
DNNs are widely used to perform these tasks due to their high accuracy and adaptability. The training of model parameters requires massive amounts of data that can only be achieved with powerful hardware and adequate time. Additionally, since neural networks are widely used, it is essential to research how to utilize and train them optimally.

One solution to tackle this issue is cloud computing, which undoubtedly poses real challenges to network capacity and the computing power of cloud computing infrastructures. Also, many new applications, e.g., cooperative autonomous driving, are sensitive to delay requirements that the cloud infrastructures would have difficulty meeting since they may be far from the users. 

Another solution that has become a hot topic in the computation context is edge computing. Edge computing has many benefits, such as alleviating the network traffic load as less data is exchanged with the cloud compared to the cloud-only scenario. Furthermore, services hosted at the edge can substantially reduce the delay time of data transmissions and improve the response time. The users' privacy is also enhanced as the private data is stored locally on the edge or user devices instead of cloud servers. The hierarchical computing architecture provides more reliable computation, and finally, edge computing can promote the pervasive application of DNNs.
 
Cloud computing and edge computing aren't mutually exclusive, and that's crucial to understand. In other words, edge computing complements and extends cloud computing. High processing power, giant storage, and data backup are some of the capabilities of cloud servers that edge servers lack. On the contrary, low latency, privacy assurance, and real-time processing are some of the capabilities of edge servers that cloud servers lack. The combination of these computation paradigms enables cumulating advantages of each and reduces the disadvantages.
 
% <-------- Add a paragraph about what is edge intelligence, why it is important, some of its applications, etc.

% <-------- Also complete this paragraph. Don't explain about method and only talk about why placement is important. Also add some info about partitioning so that other side of problem is emphasized. 
Our focus is on the DNN model placement in the context of edge intelligence.

\section{Related work}
Throughout this section, we review some research related to mobile edge intelligence, server selection, and model placement.


Edge intelligence (EI), the integration of mobile edge computing (MEC) and AI technologies, has recently emerged as a promising paradigm to support computation-intensive AI applications at the network edge \cite{9442308}.
An edge-based MEC network with multiple edge servers is studied in \cite{8737385} to maximize the number of computing offloading requests under edge storage, computation, and communication constraints. To cope with the unknown and fluctuating service demand,\cite{8509631} proposed an online learning algorithm to optimize spatial-temporal dynamic service placement decisions among multiple edge servers to minimize the computation delay. Considering parallel computing at both cloud and edge servers, \cite{xu2018joint} and \cite{chen2019collaborative} studied collaborative service placement and computation offloading to minimize the computation latency.

Due to resource limitations on the edge server, server selection has received considerable attention in recent years \cite{9599379}. 
Researchers have proposed techniques to achieve optimal goals when selecting servers, such as minimizing the average delay \cite{8972932},
maximizing the resources utilization efficiency \cite{8823875}, minimizing energy consumption \cite{li2020energy} and etc.

% <------ This paragraph is pretty wrong
In some researches like \cite{liu2021deep}, to deal with edge computing's challenges, they model the problem of continuous server
selection as a Markov Decision Process (MDP). The difficulty of this problem is that achieving long-term optimum requires future knowledge, such as usermobility, serverworkload, etc, which is not known a priori and they for dealing to this issue, they have proposed  Deep Reinforcement Learning (DRL). Although, this technique can reach to good solutions but it's crucial to consider that the amount of time and computations to train DRL models is considerable. Besides, the DRL model should be trained periodically thus it seems this method can't be use in many situations specially in real-time application.

\section{Methodology}
This is my method

\bibliographystyle{IEEEtran}
\bibliography{IEEEabrv,citation.bib}
\end{document}
